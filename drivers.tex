% This document is available under the Creative Commons Attribution-ShareAlike
% License; additional terms may apply. See
%   * http://creativecommons.org/licenses/by-sa/3.0/
%   * http://creativecommons.org/licenses/by-sa/3.0/legalcode
%
% Copyright 2010 Jérôme Pouiller <jezz@sysmic.org>
%

\part{Modules noyau}

\begin{frame}
  \partpage
\end{frame}

\begin{frame}
  \tableofcontents[currentpart]
\end{frame}


\section{Options principales du noyau}

\subsection{Configuration globale}

\begin{frame}[fragile=singleslide]{Configuration globale}
  \emph{General setup}:
  \begin{itemize} 
  \item \emph{Prompt for  development and/or incomplete code/drivers}:
    Débloque  les  options  de  compilation  pour  les  drivers/option
    instables (staging, etc...)
  \item  \emph{Cross-compiler  tool  prefix}  :  Affecte  la  variable
    \c{CROSS_COMPILE}
  \item   \emph{Local  version}   :   Ajoute  un   identifiant  à   la
    version. Indispensable  dans les phases  d'intégration. La version
    peut être lue dans  \file{/proc/version}. Il est aussi possible de
    faire \c{make kernelrelease} dans  un répertoire de compilation du
    noyau.
  \item  \emph{Automatically  append  version  information}  :  Ajoute
    l'identifiant git  à la version. Indispensable dans  les phases de
    développement
  \item \emph{Kernel  compression mode}: Permet de choisir  le type de
    compression.   Chaque  algorithme   a  ces  inconvénients  et  ses
    intérêts.
    \note[item]{cf. \url{http://free-electrons.com/blog/lzo-kernel-compression}}
  \item \c{SWAP}: Permet de gérer un espace d'échange dur un disque
  \end{itemize}
\end{frame}

\begin{frame}[fragile=singleslide]{Configuration globale}
  \begin{itemize} 
  \item  \c{SYSVIPC} et  \c{MQUEUE}:  Communication inter-processus
    définis par Posix
  \item \c{IKCONFIG}: Embarque le \file{.config} dans le noyau 
  \item  \c{EXPERT} et  \c{EMBEDDED} Débloque  les  options permettant
    principalement  de réduire la  taille du  noyau en  supprimant des
    modules importants
  \item \c{CC_OPTIMIZE_FOR_SIZE}: Compile avec \c{-Os}
  \item \c{KPROBES},  \c{PERF_EVENTS}, \c{PROFILING}, \c{GCOV_KERNEL}:
    Active les différentes instrumentations du noyau
  \end{itemize} 
\end{frame}

\begin{frame}[fragile=singleslide]{Les périphériques de block}
  \c{MODULES}: Active la gestion des modules
  \\[2ex]
  \c{BLOCK}: Il est possible  de désactiver la gestion des périphérique
  de block si votre système n'utilise que de la mémoire flash.
  \begin{itemize} 
  \item \emph{IO  Schedulers}: Permet de choisir  un ordonnanceur d'E/S
    différent de celui proposé en standard
  \end{itemize} 
  \emph{System type}:
  \begin{itemize} 
  \item Permet de choisir le type d'architecture et de chipset
  \item Il est  possible de désactiver certains cache  lors des phases
    de développement
  \item Vous trouverez aussi dans  ce menu les options relative au jeu
    d'instructions accepté
  \end{itemize}
\end{frame}

\subsection{Fonctionnalités du noyau}

\begin{frame}[fragile=singleslide]{Options de l'horloges}
  \emph{Kernel features}
  \begin{itemize} 
  \item \c{HZ} (pas sur ARM): Définit l'intervalle de réordonnancement
    de l'ordonnanceur.   Plus cette valeur est  forte, plus l'overhead
    introduit par le changement de  contexte est important et plus les
    temps de réponses des tâches sont courts
  \item \c{NO_HZ}: Permet de rendre la période de réordonnancement des
    tâches  dynamique.   Devrait  permettre  un  léger   gain  de  CPU
    (finalement  négligeable avec  l'ordonnanceur  en $o(1)$).  Permet
    surtout de gagner en consommation électrique.
  \item   \c{HIGH_RES_TIMER}:  Gère  les   timers  avec   une  horloge
    différente de  l'ordonnanceur (l'horloge  est alors géré  comme un
    périphérique  à  part).    Permet  d'obtenir  une  bien  meilleure
    précision sur les mesure de  temps, à condition que votre matériel
    possède une horloge \emph{HighRes}.
  \end{itemize}
\end{frame}  

\begin{frame}[fragile=singleslide]{Options de l'ordonnanceur}
  \begin{itemize} 
  \item  \emph{Preemption Model}:  Permet d'activer  la  préemption du
    noyau. Le pire  temps réponse sont améliorés, mais  le temps moyen
    est généralement  moins bon.  Un noyau  préemptif stresse beaucoup
    plus  de  code.  Ne  pas  activer  si  vous utilisez  des  drivers
    extérieur non garanti pour cette option.
  \item \c{RT_PREEMPT} (sur certaines architectures seulement): Permet
    de threader  les IRQ  et ainsi de  remplacer les spinlock  par des
    mutex.  Ajoute un protocole  d'héritage de priorité aux mutex.  Le
    kernel devient alors totalement  préemptif.  A n'utilisez que lors
    d'application  temps   réelle.   Etudiez  des   solutions  à  base
    d'hyperviseurs.
  \item Ne confondez pas la préemption du noyau avec la préemption des
    tâches utilisateur.
  \end{itemize}
\end{frame}  

\begin{frame}[fragile=singleslide]{Option de gestion de la mémoire}
  \begin{itemize} 
  \item \c{EABI},  \c{OABI}, etc...  : Différentes format  d'appel des
    fonctions. Spécifique à ARM (mais très important)
  \item  \emph{Memory Model}: Permet  de gérer  les futurs  systèmes à
    mémoire asymétriques entre les CPU
  \item  \c{COMPACTION}:  Permet de  compresser  les  page de  mémoire
    plutôt  que les mettre  en swap.  Particulièrement utile  dans les
    systèmes sans swap !
  \item  \c{KSM}: Permet  de  fusionner les  page mémoire  identiques.
    Uniquement   utile   avec   des   machines   virtuelles   ou   des
    chroot. Sinon, les  noyau sait que le fichier  est déjà en mémoire
    et ne duplique pas la page
  \end{itemize} 
\end{frame}

\subsection{Les options de boot}

\begin{frame}[fragile=singleslide]{Configuration du boot et du FPE}
  \emph{Boot options}:
  \begin{itemize}
  \item \emph{Flattened Device  Tree} : Utilise \emph{OpenFirmware}, le
    nouveau format de  description matériel appelé aussi \emph{Flatten
      Device Tree}
  \item  \emph{Default kernel  command string}:  Permet de  passer des
    paramètres  par défaut  au noyau  (nous verrons  cela un  peu plus
    loin)
  \item \emph{boot loader address}:  Permettent de démarrer le noyau à
    partir d'une ROM, d'une MMC, etc...
  \item \emph{Kernel Execute-In-Place  from ROM}: Permet d'exécuter un
    noyau non compressé à partir d'une ROM
  \end{itemize}
  \emph{Floating point emulation}: Si  une instruction sur des nombres
  à virgule flottante est rencontrée  et ne peut pas être exécutée, le
  noyau peut alors émuler l'instruction (voir aussi \c{-msoft-float})
\end{frame}

\subsection{Le réseau}

\begin{frame}[fragile=singleslide]{Configuration réseau}
  \emph{Networking}:
  \begin{itemize}
  \item Possibilité  d'activer les innombrables  protocoles réseaux de
    niveaux 1, 2 et 3
  \item  \emph{Network options} : Beaucoup  de fonctionnalité  réseau :
    client dhcp,  bootp, rarp, ipv6, ipsec, les  protocole de routage,
    gestion de QoS, support des VLAN, du multicast,
  \item   \emph{Unix  domain  sockets}   :  Les   sockets  \emph{UNIX}
    (cf. sortie de \c{netstat})
  \item   \emph{TCP/IP   networking}  :   Les   sockets  bien   connue
    \emph{TCP/IP}
  \item  \emph{Netfilter}  :  Le  firewall  de  Linux.   D'innombrable
    options.  Permet l'utilisation d'iptables si l'option \c{IPTABLES}
    est active.
  \end{itemize} 
\end{frame}

\subsection{Les systèmes de fichiers}

\begin{frame}[fragile=singleslide]{Configuration des systèmes de fichiers}
  \emph{File systems}:
  \begin{itemize}
  \item   \emph{Second   extended},   \emph{Ext3  journalling   file},
    \emph{The Extended 4 filesystem}: Le file system standard de Linux
  \item \emph{FUSE}: Permet de  développer des systèmes de fichiers en
    espace utilisateur
  \item \emph{Pseudo  filesystems} Systèmes de  fichiers sans supports
    physiques
    \begin{itemize} 
    \item \c{TMPFS}: File system  volatile en RAM.  Très utilisé avec
      des système en flash vu que  l'accès à la Flash est coûteux en
      temps et destructeur pour la flash
    \item \c{SYSFS} et \c{PROC_FS}:  Permettent au noyau d'exporter un
      certain  nombre de  donnée interne  vers le  userland.  Beaucoup
      d'outils  système tirent  lors informations  de ces  systèmes de
      fichiers.  Ils  doivent être montés dans  \c{/sys} et \c{/proc}.
      \c{/proc} est  plutôt orienté  processus alors que  \c{/sys} est
      orienté modules et paramétrage du noyau.
    \end{itemize} 
  \end{itemize}
\end{frame}

\begin{frame}[fragile=singleslide]{Configuration des systèmes de fichiers}
  \begin{itemize}
  \item  \emph{Miscellaneous  filesystems}  Contient des  systèmes  de
    fichiers spécifiques
    \begin{itemize} 
    \item  \emph{eCrypt filesystem layer}  : Gestion  transparent d'un
      file system codé
    \item \emph{Journalling  Flash File  System v2} :  Spécialisé pour
      les  Flash avec  gestion de  l'écriture uniforme,  des \emph{bad
        blocks} et des \emph{erase blocks}.
    \item \emph{Compressed ROM file  system}: Spécialisé pour ROM sans
      accès en écriture.
    \item  \emph{Squashed   file  system}:  Idem   \emph{cramfs}  mais
      fortement compressé
    \end{itemize} 
  \item \emph{Network File Systems}
    \begin{itemize}
    \item \emph{NFS client support}  : File system sur ethernet.  Très
      utilisé dans l'embarqué durant les phases de développement
    \item \emph{Root file system on  NFS}: Permet de démarrer le noyau
      sur une partition NFS
    \end{itemize} 
  \end{itemize} 
\end{frame}

\subsection{Les drivers}

\begin{frame}[fragile=singleslide]{Configuration des Drivers}
  \emph{Device Drivers} Des centaines de drivers. Notons:
  \begin{itemize}
  \item   \emph{path to uevent helper}:  Le  programme   apellé  lorsqu'un
    nouveau            périphérique            est           détecté
    (cf.              \c{/proc/sys/kernel/hotplug}             et
    \c{/sys/kernel/uevent_helper})
  \item  \emph{Maintain a devtmpfs  filesystem to  mount at  /dev}: Un
    tmpfs  spécifique  pour  les  devices  automatiquement  monté  sur
    \file{/dev}.   Les  fichiers  devices sont  alors  automatiquement
    créés sans l'aide d'un programme extérieur.
  \item \emph{Memory Technology Device}: Les flashs
  \item \emph{Staging drivers}: Des drivers en cours de bêta
  \end{itemize} 
\end{frame}

\subsection{Autres options}

\begin{frame}[fragile=singleslide]{Configuration du noyau}
  Mais aussi:
  \begin{itemize} 
  \item \emph{Kernel  Hacking}:  Options  concernant le  débugging  du
    noyau. % Nous y reviendrons dans la section Debug
  \item  \emph{Security  Options}:  Plusieurs framework  permettant  de
    gérer des  droits plus  fin sur les  programmes exécutés  et/ou de
    garantir l'intégrité des donnée à l'aide de TPM.
  \item   \emph{Cryptographic   API}:   Fonctions   de   cryptographies
    sélectionnées     automatiquement     par     d'autres     modules
    (particulièrement les protocoles réseaux)
  \item  \emph{Library routines}: Idem  \emph{Cryptographic API}  mais
    avec principalement des calculs de checksum.
  \end{itemize} 
\end{frame}

\section{Création de modules}

\subsection{Template}

\begin{frame}[fragile=singleslide]{Les modules noyau}{my\_module}
  \lstinputlisting[language=c,firstline=16]{samples/mod0_min/mod0_min.c}
\end{frame}

\begin{frame}[fragile=singleslide]{Quelques macros de base}
  Ces macros  permettent de placer  des informations sur  des symboles
  particulier dans module;
  \begin{itemize} 
  \item Déclare la fonction à apeller lors du chargement du module
    \begin{lstlisting} 
module_init
    \end{lstlisting} 
  \item Déclare la fonction à appeller lors du déchargement du modules
    \begin{lstlisting} 
module_exit
    \end{lstlisting} 
  \item Déclare un paramètre
    \begin{lstlisting}
module_param
    \end{lstlisting}
  \item Documente un paramètre pour modinfo
    \begin{lstlisting}
MODULE_PARM_DESC
    \end{lstlisting}
  \item Déclare un auteur du fichier. Peut apparaitre plusieurs fois.
    \begin{lstlisting}
MODULE_AUTHOR
    \end{lstlisting}
  \end{itemize}
\end{frame}
\begin{frame}[fragile=singleslide]{Quelques macros de base}
  \begin{itemize} 
  \item Description du modules
    \begin{lstlisting}
MODULE_DESCRIPTION
    \end{lstlisting}
  \item License. Indispensable
    \begin{lstlisting}
MODULE_LICENSE
    \end{lstlisting}
  \item Version du module
    \begin{lstlisting}
MODULE_VERSION
    \end{lstlisting}
  \item Rend le symbole visible par les autres modules.  Il sera alors
    pris en compte dans le calcul des dépendances de symboles.
    \begin{lstlisting}
EXPORT_SYMBOL
    \end{lstlisting} 
  \item Idem \cmd{EXPORT\_SYMBOL} mais  ne permet sont utilisation que
    pour les modules GPL
    \begin{lstlisting}
EXPORT_SYMBOL_GPL
    \end{lstlisting} 
  \end{itemize}
\end{frame}

\subsection{Licences}

\begin{frame}[fragile=singleslide]{Parlons des licenses}
  Le noyau est sous license GPL. Néanmoins, le débat est ouvert sur la
  possibilité  qu'un module  propriétaire puisse  se linker  avec.  Le
  débat n'est pas  tranché. Le noyau laisse la  posisbilité à l'auteur
  d'exporter   ses   modules   avec   \verb+EXPORT_SYMBOL+   ou   avec
  \verb+EXPORT_SYMBOL_GPL+.
\\[2ex]
  Si vous développez un module Propriétaire, vous n'aurez pas accès à
  toute l'API du noyau (environ 90\% seulement).
\\[2ex]
  Il est néanmoins possible de contourner le problème en utilisant un
  module intermédiaire comme proxy logiciel.
\end{frame}

\begin{frame}[fragile=singleslide]{Les modules noyau}{my\_module}
  \lstinputlisting[language=c,lastline=14]{samples/mod0_min/mod0_min.c}
\end{frame}

\subsection{Compilation externe}

\begin{frame}[fragile=singleslide]{Les modules noyau}{my\_module}
  Makefile:
  \begin{lstlisting}
obj-m := my_module.o  
  \end{lstlisting}
  Puis, on appelle:
  \begin{lstlisting}
host$ KDIR=/lib/modules/$(uname -r)/build
host$ make -C $KDIR ARCH=arm SUBDIRS=$(pwd) modules
  \end{lstlisting} % $
  Pour améliorer le processus, on ajoute ces lignes dans le Makefile:
  \begin{lstlisting}
KDIR ?= /lib/modules/$(shell uname -r)/build

default:
        $(MAKE) -C $(KDIR) SUBDIRS=$(shell pwd) modules
  \end{lstlisting}
  et on appelle
  \begin{lstlisting}
host$ make ARCH=arm CROSS_COMPILE=arm-linux- KDIR=../linux-2.6/usb-a9260 
  \end{lstlisting} % $
\end{frame}

\subsection{Compilation interne}

\begin{frame}[fragile=singleslide]{Compilation avec Kbuild}
  Fichier \file{Makefile} à l'intérieur de l'arborescence noyau:
  \begin{lstlisting}
obj-$(MY_COMPILE_OPTION) := my_module.o  
  \end{lstlisting} % $
  \lstinline+$(MY_COMPILE_OPTION)+ sera remplacé par :
  \begin{itemize}
  \item ø: Non compilé
  \item m: compilé en module
  \item y: compilé en statique
  \end{itemize}
  Fichier \file{Kconfig}:
  \lstinputlisting[language=]{samples/mod0_min/Kconfig}
\end{frame} 

\begin{frame}[fragile=singleslide]{Utilisation de Kconfig}
  Chaque entrée \c{config} prend comme attribut:
  \begin{itemize} 
  \item Son type et sa description en une ligne:
    \begin{itemize}
    \item \c{tristate}, le plus  classique pouvant prendre les valeurs
     ø, m et y
    \item \c{bool} pouvant prendre seulement les valeurs n et y
    \item \c{int} prennant un nombre
    \item \c{string} prennant une chaîne
    \end{itemize} 
  \item \c{default} Sa valeur par défaut
  \item  \c{depends on} L'option  n'apparait si  l'expression suivante
    est vraie.  Il est possible de spécifier  des conditions complexes
    avec les opérateurs \c{&&}, \c{||}, \c{=} et \c{\!=}
  \item \c{select}  Active automatiquement les options  en argument si
    l'option est activée
  \item   \c{help}  Description   détaillée  de   l'option.   Si  votre
    description ne  tient pas en moins  de 20 lignes,  pensez à écrire
    une documentation dans \file{Documentation} et à y faire référence
  \end{itemize} 
\end{frame}

\begin{frame}[fragile=singleslide]{Utilisation de Kconfig}
  Il est aussi possible:
  \begin{itemize} 
  \item D'inclure d'autres fichiers avec \c{source}
  \item De déclarer un sous menu avec \c{menu}
  \item  De demander  un choix  parmis un  nombre fini  d'options avec
    \c{choice}
  \end{itemize} 
\end{frame}

\begin{frame}[fragile=singleslide]{Gérer les modules}
  \begin{itemize} 
  \item Avoir des informations sur le module
    \begin{lstlisting}
host$ modinfo my_module.ko
    \end{lstlisting} %$
  \item Charger un module
    \begin{lstlisting}
target% insmod my_module.ko
    \end{lstlisting} %$
  \item Décharger un module
    \begin{lstlisting}
target% rmmod my_module
    \end{lstlisting}%$
  \item Afficher le buffer de log du kernel
    \begin{lstlisting}
target$ dmesg
    \end{lstlisting} %$
  \item Charger/décharger un module correctement installé/indexé
    \begin{lstlisting}
target% modprobe my_module
target% modprobe -r my_module
    \end{lstlisting} %$
  \item Mettre à jour le fichier de dépendance
    \begin{lstlisting} 
target% depmod
    \end{lstlisting} %$
  \end{itemize}
\end{frame}

\begin{frame}[fragile=singleslide]{Paramètres}
  Il est possible de passer des paramètres aux modules:
  \begin{lstlisting}
target$ modinfo my_module.ko 
target% insmod my_module.ko param=2  
  \end{lstlisting} %$ 
  Nous devons déclarer le paramètre à l'aide de la macro \texttt{module\_param}. 
  \begin{lstlisting}[language=c]
#include <linux/moduleparam.h>
module_param(param, int, 0644);
  \end{lstlisting} 
  La paramètre doit évidement être alloué:
  \begin{lstlisting}[language=c]
static int param = 0;
  \end{lstlisting} 
  Il est fortement recommandé de documenter le paramètre
  \begin{lstlisting}[language=c]
MODULE_PARM_DESC(param, "Display this value at loading and unloading");
  \end{lstlisting} 
\end{frame}

\begin{frame}[fragile=singleslide]{\texttt{/sys}}
  Etudions \file{/sys}
  \begin{itemize}
  \item                       \file{/sys/module/my\_module/parameters}:
    paramètres. Modifiable si déclaré modifiables
  \item \file{/sys/module/my\_module/sections}:  des info sur  la zone
    de chargement
  \end{itemize}
\end{frame}

\begin{frame}[fragile=singleslide]{Les modules noyau}{my\_module}
  Quelques conseils:
  \begin{itemize} 
  \item  Pour mettre un code à la norme:
    \begin{lstlisting}
host$ apt-get indent
host$ scripts/Lindent my_module.c
host$ scripts/cleanfile my_module.c
    \end{lstlisting} %$
  \item Voir \file{Documentation/CodingStyle}
  \item Sauvez l'espace de nom du noyau: utilisez des static !
  \item Votre code doit être réentrant
  \end{itemize} 
\end{frame}

% TODO: A revoir
\section{Notre premier device}

\begin{frame}[fragile=singleslide]{Communiquer avec le noyau}{char\_dev}
  \begin{itemize}
  \item Voie classique pour un driver pour communiquer avec le noyau
  \item Major, Minor
    \begin{lstlisting}
target% mknod my_chardev c <major> <minor>
    \end{lstlisting}
  \item Communication en écrivant/lisant des données
    \begin{lstlisting}
target% insmod my_chardev.ko
target% echo toto >  my_dhardev
target% cat my_chardev
    \end{lstlisting}
  \item Possibilité de faire un peu de configuration par les ioctl
  \end{itemize}
\end{frame}

\begin{frame}[fragile=singleslide]{Communiquer avec le noyau}{char\_dev}
  Exo: Faire un pipe avec un buffer:
  \begin{lstlisting}
target% insmod my_chardev.ko
target% echo toto >  my_dhardev
target% cat my_chardev
toto
  \end{lstlisting}
  \begin{itemize}
  \item \verb+copy_to_user+, \verb+copy_from_user+
  \item \verb+kcalloc+, \verb+kfree+
  \item \verb+register_chrdev+, \verb+unregister_chrdev_region+
  \item   \verb+mutex_lock+,  \verb+mutex_unlock+,  \verb+mutex_init+,
    \verb+mutex_destroy+
  \item \verb+memmove+
  \end{itemize}
\end{frame}

\begin{frame}[fragile=singleslide]{Communiquer avec le noyau}{char\_dev}
   Vérifions votre comportement:
   \begin{lstlisting}
target% insmod my_chrdev.ko
target% mknod my_chrdev c 251 0
target$ for i in {1..64}; echo "$i " > my_chrdev
target$ cat my_chrdev
target% rmmod my_chrdev
target% insmod my_chrdev.ko buf_size=2
target$ echo foo > /dev/my_chrdev
target$ cat /dev/my_chrdev
target% rmmod my_chrdev
target% insmod my_chrdev.ko buf_size=-1
    \end{lstlisting} % $
\end{frame}
 
% \begin{frame}{Les ioctls}
%   C'est un appel système qui permet de faire passer une structure
%   quelconque à un device.\\
%   Pour appeller un ioctl, il faut un device, le numéro de l'IOCTL et
%   l'arguments.\\
%   Les  numéro d'IOCTL  se  décode ainsi  (attention,  ca n'est  qu'une
%   norme, et elle a ses exeception, principalement powerpc):
% %   \begin{lstlisting}
% %  bits    meaning
% %  31-30  00 - no parameters: uses _IO macro
% %         10 - read: _IOR
% %         01 - write: _IOW
% %         11 - read/write: _IOWR
% % 
% %  29-16  size of arguments
% % 
% %  15-8   ascii character supposedly
% %         unique to each driver
% % 
% %  7-0    function #
% % 
% % 
% % So for example 0x82187201 is a read with arg length of 0x218,
% % character 'r' function 1. Grepping the source reveals this is:
% % Les ioctl doivent être unique par device. Mais on préfère qu'il soit unique
% % sur tout le système.
% % 
% % Les macro _IOR, _IOW, _IORW et _IO nous aident:
% % #define CHANGE_BUF_SZ _IOR(42, 1, int)
% %    \end{lstlisting}

% Aller plus loin: \file{device.h}: Implémentation d'un device complet.
% \end{frame}

\begin{frame}[fragile=singleslide]{Obtenir de la documentation}
  \begin{lstlisting}
host$ make mandocs
host% make installmandocs
  \end{lstlisting}  % $
  \begin{itemize}
    \item Paquet linux-doc: \file{/usr/share/doc/linux-doc/html/kernel-api}
    \item Pas de paquet pour ces pages de man 
    \item ou pages de man sur \url{http://tfm.cz}
  \end{itemize}
\end{frame}

% TODO: Faire un device usb-gadget/hid 

% Note
% echo none > /sys/devices/platform//leds-gpio/leds/user_led/trigger
% echo 255 > /sys/devices/platform//leds-gpio/leds/user_led/brightness

