%
% This document is available under the Creative Commons Attribution-ShareAlike
% License; additional terms may apply. See
%   * http://creativecommons.org/licenses/by-sa/3.0/
%   * http://creativecommons.org/licenses/by-sa/3.0/legalcode
%
% Created: 2012-07-28 10:50:36+02:00
% Main authors:
%     - Jérôme Pouiller <jezz@sysmic.org>
%

\section{Le shell}

\begin{frame}
  \partpage
\end{frame}

\begin{frame}
  \tableofcontents[currentpart]
\end{frame}

\begin{frame}[fragile=singleslide]{Historique}
  \begin{itemize}
  \item Mode de communication bas niveau privilègié
  \item  Lèger,  simple  à  implémenter, puissant.   Parfois  l'unique
    manière de communiquer avec le système.
  \item Le shell ``Unix'' est plus commun. Beaucoup d'autres interface
    en ligne de commande s'en inspire.
  \item Première version du shell Unix tel qu'on le connait écrite par
    Ken Thompson chez Bell Labs en 1971 (bien antérieur à Linux)
  \item Remplacé par le shell de Stephen Bourne en 1977
  \item Par ordre d'apparition: sh, csh, tcsh, ksh, bash, zsh, ash,
    dash
  \item Normalisé par la Posix 2 en 1992
  \item Fonctionne de manière plus ou moins identique sur tous les OS
    (Linux, Androïd, iOS, Windows/Cygwin, etc...)
  \item On peut faire beaucoup de chose avec la ligne de commande
  \item Il est possible de faire des script en shell. Il arrive ce
    soir le seul langage de script disponible sur le système.
  \end{itemize}
\end{frame}

\subsection{Bases}


\begin{frame}[fragile=singleslide]{Bases de shell}
  \begin{itemize}
  \item Lancer une commande (=lancer un programme)
    \begin{lstlisting}
$ ls
    \end{lstlisting} %$
  \item Séparation des arguments par des espaces
    \begin{lstlisting}
$ mkdir dir1 dir2
    \end{lstlisting} %$
  \item Conséquence: les espaces sont des caractère spéciaux en shell
  \item Les arguments sont recu par le programme par es deux arguments
    \c{argc} et \c{argv}
  \end{itemize}
\end{frame}

\begin{frame}[fragile=singleslide]{Convention}
  Par convention,  nous préfixons dans ces slides  les commandes shell
  par :
  \begin{itemize}
  \item  \verb+$+  pour les  commandes  à  éxecuter par  l'utilisateur
    normal
  \item \verb+%+ pour les commande à executer par root
  \item \verb+>+ pour les commandes non-shell
  \end{itemize}
\end{frame}

\begin{frame}[fragile=singleslide]{Les argument optionnels}
  \begin{itemize}
  \item Souvent les arguments optionnels commence par '-'
  \item Les options peuvent avoir des arguments
  \item On distingue les options longues commencant par '-\--'
    \begin{lstlisting}
$ ls --all
$ ls --sort=time
$ ls --sort time
    \end{lstlisting}
  \item ... les option courte tenant sur un seul caractère et
    commancant par un simple '-' (attention tout de même aux
    exceptions)
    \begin{lstlisting}
$ ls -l -a
$ ls
    \end{lstlisting}
  \item Il est possible de concaténer les option courtes
    \begin{lstlisting}
$ ls -la
    \end{lstlisting} %$
    \item Les options ne sont \emph{normalement} pas dépendante de leur emplacement
    \begin{lstlisting}
$ ls -la
    \end{lstlisting} %$
  \item Ce principe est normalisé car toutes ces commandes utilisent
    la fonction Posix getopt (mais rien ne le garanti)
  \end{itemize}
\end{frame}

\begin{frame}[fragile=singleslide]{La documentation}
  \begin{itemize}
  \item   \cmd{man COMMAND} permet d'accèder à la documentation de la commande
  \item Les pages de man sont divisée en 9 sections (d'après \emph{man(1)})
    \begin{enumerate}
    \item Executable programs or shell commands
    \item System calls (functions provided by the kernel)
    \item Library calls (functions within program libraries)
    \item Special files (usually found in /dev)
    \item File formats and conventions (e.g. /etc/passwd)
    \item Games
    \item Miscellaneous (including macro packages and conventions),
      e.g.  man(7), groff(7)
    \item System administration commands (usually only for root)
    \item Kernel routines [Non standard]
    \end{enumerate}
  \item Une même entrée peut être présente dans plusieurs section, il
    possible de préciser la section en la placant en argument avant la
    commande:
    \begin{lstlisting}
$ man read
$ man 2 read
    \end{lstlisting}
  \item Les références des pages de man sont donnés avec le numéro de
    section entre parenthèses.  Ainsi, \textit{wait(2)} signifie que
    vous pouvez accéder à la documentation avec la commande
    \verb+man 2 wait+.
  \item \cmd{man -l}  permet d'afficher un fichier ``local''
  \item \c{man man} pour plus d'information
  \end{itemize}
\end{frame}

\begin{frame}[fragile=singleslide]{Les chemins}
  Il est possible d'utiliser des chemins:
  \begin{itemize}
  \item absolus, commencant par un \c{/}
    \begin{lstlisting}
$ mkdir /tmp/tete
    \end{lstlisting}
  \item relatifs, commencant par un autre caractère
    \begin{lstlisting}
$ mkdir ../../tmp/titi
    \end{lstlisting}
  \end{itemize}
  Les chemins relatifs, s'interprête à partir du \emph{répertoire
    courant} (lié au processus actuel et hérité par les processus
  fils).
  \begin{itemize}
  \item \c{pwd} affiche le répertoire courant
  \item \c{cd} modifie le répertoire courant
  \item Dans un chemin, "." correspond au répertoire courant
  \item ... \cmd{mkdir foo} est identique à \cmd{mkdir ././foo}
  \item ".." correspond au répertoire parent
  \item Beaucoup de commande prennant en parametre un répertoire
    utilise le répertoire courant si le paramètre n'est pas spécifié
    (ex: \c{ls})
 \item  cf. \emph{path\_resolution(7)}
  \end{itemize}
Note: Les fichiers commencant par \c{.} sont considérés comme des fichiers cachés
\end{frame}

\begin{frame}[fragile=singleslide]{Le PATH}
  \begin{itemize}
  \item Normalisées par Posix, plus ou moins regroupée dans un projet
    nommé coreutils
  \item Une commande est recherchée dans la variable \c{$PATH}
  \item ... par défaut: /bin /sbin /usr/bin et /usr/sbin
  \item Si on spécifie le chemin (la commande contient \c{/}),
    \c{$PATH} n'est pas utilisé
  \item Par conséquent, pour lancer une binaire dans le répertoire
    courant:\c{./a.out}
  \item Ajouter \c{.} dans \c{$PATH} est une mauvaise pratique
  \item Mecanisme géré par la fonction Posix execvpe(3)
  \item Certaine commande ont un comportement différent suivant
    \c{argv[0]} (exemple: \c{test} et \c{[})
  \end{itemize}
\end{frame}

\begin{frame}[fragile=singleslide]{Le contenu des fichiers}
  \begin{itemize}
    \item Ne pas oublier qu'un fichier n'est qu'un vecteur d'octet
    \item Les fichiers dit "texte" ont simplement la particularité de
      n'avoir que des octets supérieurs à 0x20 (et les octet > 0x7F
      s'interprete suivant la région)
    \item Le principe de "type" de fichier est finalement assez flou.
    \item \c{file} permet de repérer le format des fichier
    \item L'utilisation d'une norme de nommage ou d'une extention peut
      aussi aider, mais ca n'est pas une pratique native sur la
      plupart des OS.
    \end{itemize}
\end{frame}

\begin{frame}[fragile=singleslide]{Les file descriptor}
  \begin{itemize}
  \item Lorsqu'un programme souhaite accèder à un fichier, il va
    utiliser la fonction Posix \c{open(3)} qui lui retourne un nombre
    appellé \emph{file descriptor}.
  \item Il s'agit d'un identifiant pour une structure dans l'OS.
  \item Le file descriptor peut être passé à d'autre fonctions du
    système comme \c{read(3)} ou \c{write(3)}.
  \item Nous verrons plus tard que le concept de file descriptor va
    plus loin.
  \item Les couche basses de l'OS ouvre automaitquement trois file
    descriptor lors qu'un programme est lancé:
    \begin{itemize}
    \item Entrée standard (numéro 0), accessible en
      lecture. Normalement au clavier.
    \item Sortie standard (numéro 1), accessible en
      écriture. Normalement relié à l'écran
    \item Sortie d'erreur (numéro 2), accéssible en
      écriture. Normalement relié à l'écran
    \end{itemize}
  \end{itemize}
\end{frame}

\begin{frame}[fragile=singleslide]{Les redirections}
  Il est possible de demander au shell de rediriger les entrée est les
  sortie d'une commande avec les metacaractère \c{<} \c{>} et \c{|}:
  \begin{itemize}
  \item Commande standard:
    \begin{lstlisting}
$ echo foo
    \end{lstlisting}
  \item Sortie standard vers un fichier
    \begin{lstlisting}
$ echo foo > file
    \end{lstlisting}
  \item Un fichier vers l'entrée standard
    \begin{lstlisting}
$ cat -n < file
    \end{lstlisting} %$
  \item Sortie d'erreur vers un fichier
    \begin{lstlisting}
$ ls toto 2> file
    \end{lstlisting} %$
  \end{itemize}
\end{frame}

\begin{frame}[fragile=singleslide]{Les Redirections}
  \begin{itemize}
  \item Sortie standard d'une commande vers l'entrée d'une autre
    \begin{lstlisting}
$ echo bar foo | wc
    \end{lstlisting}
  \item Couplage des redirections
    \begin{lstlisting}
$ cat -n < file1 | wc > file3
    \end{lstlisting} %$
  \item L'espace n'est pas obligatoire et les redirections ne sont pas
    forcement à la fin de la ligne
    \begin{lstlisting}
$ >file2 cat<file1 -n
    \end{lstlisting} %$
  \item Certaine commande detecte que la sortie est redirigée et se
    comporte différement
    \begin{lstlisting}
$ ls
$ ls | cat -n
$ ls > file
    \end{lstlisting}
  \end{itemize}
\end{frame}

\subsection{Les variables}

\begin{frame}[fragile=singleslide]{Les variables locales}
  \begin{itemize}
  \item Affectation:
    \begin{lstlisting}
$ FOO=foo
    \end{lstlisting}
  \item En shell, l'espace est un metacaractère, donc, ces
commandes son fausses:
\begin{lstlisting}
$ FOO = foo
$ FOO=foo bar
\end{lstlisting}
\item Il est possible de les concaténer avec \c{+=}
    \begin{lstlisting}
$ FOO+=bar
    \end{lstlisting}
  \item La syntaxe \c{$\{VAR\}} permet de récupérer le contenu d'une
    variable. Elle peut-être abbregée \c{$VAR} elle suivit d'un
    caractère non-alpha-numérique
     \begin{lstlisting}
$ echo ${FOO}
$ echo $FOO
$ echo ${FOO}_bar $FOO_bar
     \end{lstlisting}
     \item Sous zsh, \c{vared} permet d'éditer intéractivement une variable
   \end{itemize}
 \end{frame}

\begin{frame}[fragile=singleslide]{Les variables d'environnement}
  \begin{itemize}
  \item Tous les processus possède un ensemble de variable d'envionement.
  \item Elle se trouvent dans l'espace mémoire du processus (cf. \c{environ(7)})
  \item On y accéder facilement avec \c{getenv(3)} et \c{setenv(3)}
  \item Par défaut, un processus hérite de l'environnement de son
    parent (cf. \c{exev(3)})
  \item Un programme peut modifier son comportement en fonction du
    contenu de l'environnement
  \item \c{export} liste les variable d'environnement
  \item \c{export VAR} transforme une variable locale en variable
    d'environement, ou instancie la variable.
  \item Il est possible de lancer une commande avec une varibale
    d'envionnement particulière:
    \begin{lstlisting}
$ LANG=fr_FR.utf8 ls non-existant
ls: impossible d'accéder à non-existant: Aucun fichier ou dossier de ce type
    \end{lstlisting}
  \item ... ou en utilisant la commande \c{env}, qui possède plus d'options
    \begin{lstlisting}
$ env LANG=fr_FR.utf8 ls non-existant
ls: impossible d'accéder à non-existant: Aucun fichier ou dossier de ce type
    \end{lstlisting}
  \item Les variables d'environnement auront un impact sur tous les
    sous-processus lancés
  \item Leur fonctionnement est très différent des variable shell,
    mais elle sont gérée avec la même syntaxe
  \end{itemize}
  Les variables d'environnement importantes:
  \begin{itemize}
  \item \c{PATH}: les chemins ou les commandes doivent être recherchée
  \item \c{LANG}, \c{LOCALE} et \c{LC_*}: les informtion
    d'internationnalisation
  \item \c{DISPLAY}: l'addresse du serveur d'affichage pour les
    commande graphique
  \item \c{TERM}: le type de terminal utilisé (nécessaire pour le bon
    affichage des couleur et des outils fenetrés)
  \item \c{LS_COLOR}: contient la configuration de coloration de la
    command \c{ls --color}
  \item \c{PAGER}, \c{EDITOR}, \c{BROWSER}: Les outils à utilisé pour
    visualiser, éditer et aller sur le web.
  \end{itemize}
\end{frame}

\begin{frame}[fragile=singleslide]{La syntaxe évoluée des variable}
  La syntax des  variables peut être plus évoluée:
  \begin{itemize}
  \item \c{$\{VAR#foo\}} ou \c{$\{VAR/foo/bar\}} pour effectuer des
    modifcation sur les variables
  \item \c{VAR=( a b c )} pour affecter un tableau
  \item \c{$\{VAR[3]\}} pour lire une value dans un tableau
  \item Dans un script ou dans une fonction shell, les variable
    \c{$1}, \c{$2}, ... correspondent aux arguments. \c{$*} et \c{$@}
    signifient ``Tous les arguments''
  \item \c{$((21 * 2))} et \c{$[43 - 1]} sont remplacé par le résultat
    de l'expression aritthmétique
  \item \c{$(echo toto)} ou \c{`echo toto`} sont remplacés par le
    résultat de la commande \c{echo toto}.
  \end{itemize}
\end{frame}

\begin{frame}[fragile=singleslide]{L'escaping}
  \begin{itemize}
  \item Sans surprise, il est possible d'échapper un caractère spécial avec \c{\\}
    \begin{lstlisting}
$ perl -e print\ \"Hello\ World\\n\"\;
    \end{lstlisting}
  \item Il est aussi possible de \emph{quoter} un argument
    \begin{itemize}
    \item Le \emph{double quote} \c{"} echappe la plupart des
      caractères sauf les variable et le \c{"} de fin:
      \begin{lstlisting}
$ perl -e "print \"Hello World\n\";"
      \end{lstlisting}
    \item Le \emph{simple quote} \c{'} escape tous les caractère
      sauf le carcatère \c{'} de fin. Il ne peux pas être echappé:
      \begin{lstlisting}
$ perl -e 'print "Hello World\n";'
      \end{lstlisting}
    \item Le \emph{backquote} n'est pas un quoting, il correspond à \c{$()}
    \end{itemize}
  \end{itemize}
\end{frame}

\begin{frame}[fragile=singleslide]{Le globbing}
  \begin{itemize}
  \item Langage composé de trois metacaractères:
    \begin{itemize}
    \item \c{\?}:n'importe quel caractère
    \item \c{*}: n'importe quel caractère répété 0 ou plusieurs fois
    \item \c{[]}: N'importe lequel des caractère compris entre les crochets
    \end{itemize}
  \item Le shell essaie de faire correspondre tous les arguments
    contenant ces metacaractère avec les fichier du répertoire courant
  \item ... il remplace ensuite le pattern par la liste des fichier correspondants
    \begin{lstlisting}
$ wc -l *.c
    \end{lstlisting}
  \item Les commandes recoivent la liste des fichiers en argument, pas le pattern
  \item ... deux exception notables: \c{find -name} et \c{dpkg -l}. Il
    est necessaire de correctement les quoter pour que le pattern soit
    effectivement transmis à la commande
  \end{itemize}
\end{frame}

\begin{frame}[fragile=singleslide]{Les expressions régulières}
  \begin{itemize}
  \item Ressemble de loin au globbing
  \item Plus de metacaractères:
    \begin{itemize}
    \item \c{.}: N'importe quel caractère
    \item \c{[]}: N'importe lequel des caractères contenu entre les crochets
    \item \c{*} Le caractère précédants répété 0 ou plusieurs fois
    \item \c{+} Le caractère précédant répété 1 ou plusieurs fois
    \item \c{\{X,Y\}} Le caractère précédant répété entre X et Y fois.
    \item \c{()} Mémorise un groupe qui peut être référencé avec \c{\\X}
    \item \c{^} \c{$} Début et fin de ligne
    \item ... voir  regex(7) pour la spécifiction complète
    \item D'un point de vu formel, les expression réguli`ere peuvent
      décrire des grammaire régulière (de Type 3 dans la hiérarchie de
      chompsky). Il est possible de démontrer qu'il est existe ne
      bijection entre l'ensemble des expressions régulière et
      l'ensemble des automates à état fini (A comparer avec bison, un
      outils capable d'exprimer des grammaire de Type 2,
      \emph{context-free}).
    \item Les deux implémentation les plus répandues sont celles de
      gnu (Gnu Regular Expression Processor) et de perl. Il peut
      exister des différence entre ces implémentations.
      \begin{lstlisting}
$ echo a123b45 | grep -E '^a.*b.*$'
a123b45
$ echo a1111b1 | grep -E '^t(.)*t\1$'
a1111b1
      \end{lstlisting}
    \item Beaucoup de commande prennent des expression régulière en
      entrée. Attention à ce qu'elle ne soit pas interprétée comme du
      globbing par le shell
    \item Les expression régulière, c'est bon, mangez-en.
    \end{itemize}
  \end{itemize}
\end{frame}

\begin{frame}[fragile=singleslide]{Le terminal}
  \begin{itemize}
    \item Lourd historique
    \item Il existe(ait) énormément de terminaux différant. Chaque
      terminal à ses spécificitée:
      \begin{itemize}
      \item caractère permettant le retour à la ligne,
      \item possibilité de déplacer le curseur,
      \item possibilité de souligner,
      \item de metter des caractères en gras,
      \item terminaux en couleur,
      \item séquence retournée par les touches spéciales (backspace,
        delete, flèche, touches de fonction, composition de
        touches,...)
      \end{itemize}
    \item Le minitel est(était) un terminal compatible VT102
    \item Les temrinaux virtuel choisissent quelle norme ils
      implémente (u peuvent en faire une nouvelle)
    \item La variable d'environnement \c{$TERM} indique aux commandes
      le type de terminal. De nos jours, la valeur XTERM est de loin
      la plus répandue.
    \item \c{<Ctrl+V>} permet d'afficher la séquence de touche recue
      par le shell sans l'interpréter
    \item Les couleurs se font avec des séquence d'échappement plus ou
      moins standardisée: \c{\x1B[32m} pour le rouge, etc...
    \item La bibliothèque readline, très largement utilisée gère tout
      cet aspect. Readline inclut une base de donnée des
      fonctionnalité de tous les terminaux existant.
  \end{itemize}
\end{frame}

\begin{frame}[fragile=singleslide]{Outils dérivés}
  Quelques outils pour travailler avec les expression régulières
  \begin{itemize}
  \item ed, grep, sed, awk, perl -pe: Outils de traitement de texte
    automatique
  \item ed, ex, vi, vim: Un éditeur de texte
  \end{itemize}
\end{frame}

\begin{frame}[fragile=singleslide]{Jobs control}
  \begin{itemize}
  \item Il est possible de travailler avec plusieurs commande en parallèle
  \item \c{<CTRL+Z>} suspend la commande courante. En fait, c'est le
    noyau qui transforme cette combinaison de touche en signal et qui
    suspend le programme si celui-ci lui permet.
  \item \c{fg} continue l'éxecution de la commande en \emph{foreground}
  \item \c{bg} continue l'éxecution de la commande en \emph{background}
  \item Il est possible de lancer une commande directement en
    backgound en la terminant pas \c{&}
  \item \c{jobs} liste les jobs en cours d'éxection
  \end{itemize}
  \begin{lstlisting}
$ cp -r bigdir newdir
<CTRL+Z>
$ sleep 100 &
$ jobs
[1]  - suspended  cp -r bigdir newdir
[2]  + running    sleep 100
  \end{lstlisting}
\end{frame}

\begin{frame}[fragile=singleslide]{Ecrire des scripts}
  \begin{itemize}
  \item Les scripts possèdent la meme syntaxe que la ligne de commande
  \item Commencent par le chemin de l'interpréteur prefixé de \c{#!}
    \begin{lstlisting}
#!/bin/sh
#!/usr/bin/perl
#!/bin/sed
#!/usr/bin/make -f
    \end{lstlisting}
  \item L'OS appelle l'interpreteur et passer le fichier de script et
    ses arguments en paramètre. (Essayez avec votre propre
    application)
  \item Il est possible de les lancer en les passant directement à
    l'interpreteur de commande
  \item Doivent avoir les droits en execution
    \begin{lstlisting}
$ bash script.sh
$ chmod +x script.sh
$ ./script.sh
    \end{lstlisting} %$
  \item Il est bien sur possible d'appeller d'autres scripts
  \item Il est possible de sourcer d'autres script (!= appeller)
    \begin{lstlisting}
source lib.sh
. lib.sh
    \end{lstlisting}
  \item Il est possible de déclarer des fonctions
    \begin{lstlisting}
function bar {
}
foo() {
}
    \end{lstlisting}
  \item Il est possible de séparer deux ommande par \emph{\;}
  \item La commande \emph{test(1)} ou \c{\[(1)} permet d'nterpreter des condition
  \item Il existe aussi des structure de controle:
    \begin{lstlisting}
if test -n $VAR; then
  echo '$VAR exist'
else
  echo '$VAR does not exist'
fi
while [ -z $VAR ]; do
  VAR+=$(cat file)
done
    \end{lstlisting}
  \end{itemize}
\end{frame}

\begin{frame}[fragile=singleslide]{Quelques derniers trucs}
  \begin{itemize}
  \item  En début d'argument \c{\~/} sera remplacé par Le chemin de votre \emph{home}
  \item La syntaxe \c{a\{1,2\}b} sera remplacé par \c{a1b a2b}. Ca
    n'est pas du globbing, car ca ne matche pas avec le répertoire
    courant
  \item Complétion
    \begin{lstlisting}
$ cd /h<TAB>/j<TAB>/c<TAB>
    \end{lstlisting}
  \item Les shell modernent sont capables de faire de la completion avancée
    \begin{lstlisting}
$ man l<TAB>
    \end{lstlisting}
  \item Alias
    \begin{lstlisting}
$ alias ll="ls -l --color=auto"
$ alias cp="cp -i"
$ alias mv="mv -i"
$ alias rm="rm --one-file-system"
    \end{lstlisting} %$
  \item Il est possible de mettre des commande dans \c{\~/.bashrc} ou
    \c{\~/.zshrc} qui seront éxecutée à chaque démarrage du shell.
  \item Man de la commande courante (sous Zsh uniquement)
    \begin{lstlisting}
$ rm -<M-h>
    \end{lstlisting} %$
  \end{itemize}
\end{frame}

\begin{frame}[fragile=singleslide]{Travailler réseau en deux mots}
  \begin{itemize}
  \item \c{/sbin/ifconfig -a} donne la configuration des cartes réseaux
  \item \c{lo} correspond à la carte réseau virtuelle de localhost
  \item \c{route -n} affiche la table de routage du système (kernel)
  \item \c{/etc/resolv.conf} et \c{/etc/nsswitch.conf} contiennent la
    configuration du service de résolution de nom (dont le DNS) (libc)
  \item ... de nos jours, on utilise souvent un proxy DNS et la
    configuration du DNS se retrouve dans
    \c{/var/run/nm-dns-dnsmasq.conf}
  \item \c{iptables} gère les règle de filtrage
  \item Bien que très utilisée, ces commandes peuvent être remplacée
    par iproute2 (et sa cmmande \c{ip})
  \item \c{netstat -n} ou \c{netstat -ln} permet d'obtenir l'état des
    connexion réseau
  \end{itemize}
\end{frame}

\subsection{Un shell à distance}

\begin{frame}[fragile=singleslide]{Travailler à distance}
  Protocoles les plus utilisés:
  \begin{itemize}
  \item Telnet
    \begin{itemize}
    \item \cmd{telnetd} et \cmd{telnet}
\begin{lstlisting}
host$ telnet -l root target
target%
\end{lstlisting} %$
    \item   Pas   sécurisé,   attention   a   votre   mot   de   passe
      \note[item]{faire    une   démonstration    avec    telnetd   et
        \texttt{tcpdump  -i lo -A  port telnet}  (il faut  regarder le
        dernier caractère de chaque paquet envoyé)}
    \item \verb/<CTRL+]>/ permet d'accéder à l'interface de commande
    \end{itemize}
  \item Ssh
    \begin{itemize}
    \item \cmd{sshd} et \cmd{ssh}
\begin{lstlisting}
host$ ssh root@target
target%
\end{lstlisting} %$
    \item Sécurisé
    \item Pleins de bonus de sécurisé
    \item   Il   est   possible   de  forcer   la   déconnexion   avec
      \verb/<RET><~><.>/   et   de   suspendre  une   connexion   avec
      \verb/<RET><~><CTRL+Z>/
    \end{itemize}
  \end{itemize}
\end{frame}

\subsection{Utilisation de clefs numériques}

\begin{frame}[fragile=singleslide]{Utiliser des clef ssh}
  \begin{itemize}
  \item Possibilité de créer des clefs pour \cmd{ssh}
\begin{lstlisting}[language=sh]
host$ ssh-keygen -t dsa
\end{lstlisting} %$
  \item Toujours mettre un mot de passe sur votre clef
  \item Recopiez votre clef dans \verb+~/.ssh/authorized_keys+
\begin{lstlisting}[language=sh]
host$ ssh-copy-id root@target
\end{lstlisting} %$
  \end{itemize}
\end{frame}

\begin{frame}[fragile=singleslide]{Utiliser des clef ssh}
  \begin{itemize}
  \item Utiliser ssh-agent
\begin{lstlisting}[language=sh]
host$ ssh-agent
host$ SSH_AUTH_SOCK=/tmp/agent.3391; export SSH_AUTH_SOCK;
host$ SSH_AGENT_PID=3392; export SSH_AGENT_PID;
host$ echo Agent pid 3392;
\end{lstlisting} %$
  \item Enregistrer votre passphrase auprès de l'agent
\begin{lstlisting}[language=sh]
host$ ssh-add
\end{lstlisting} %$
  \item Forwarder votre agent
\begin{lstlisting}[language=sh]
host$ ssh -A root@target
target%
\end{lstlisting} %$
  \end{itemize}
\end{frame}

\begin{frame}[fragile=singleslide]{Forwarder des ports}
  \begin{itemize}
  \item \c{ssh -L 1080:far-far-host:80 far-host}: Une demande sur le
    port 1080 de ma machine \c{locale} est forwarder à \c{far-host} qui se
    connecte sur \c{far-far-host} sur le port 80. Très utile si
    far-far-host n'est pas directment accessible par internet.
  \item \c{ssh -R 1080:close-host:80 far-host}: Une demande sur le
    port 1080 de \c{far-host} est forwarder à ma machine \c{locale} qui se
    connecte sur \c{close-host} sur le port 80. Très utile si je dois
    acceder à \c{close-host} à partir de far-host
  \item \c{ssh -D 1080 far-host} Idem que l'option \c{-L}. En
    revanche, je peux demander à crée un tunnel vers n'importe quel
    host dynamique. On utilise le protocole socks utilisé par les
    proxy. Certaine commande intègre le gestion de ce protocole. Pour
    les autre, on peut utiliser \c{socksify}
  \item L'étape suivante est de créer une interface réseau virtuelle
    passant par ce canal et de lui affecter des route. On obtenient
    ainsi un VPN.
    \item cf. \emph{ssh(1)}
  \end{itemize}
\end{frame}

\subsection{Les utilisateurs}

\begin{frame}[fragile=singleslide]{Les utilisateurs}
  \begin{itemize}
  \item Il y a plusieurs utilisateurs sur le système. Il sont gérés
    dans \c{/etc/passwd}
  \item Il y a des groupe sur le système, gérés dans \c{/etc/group}
  \item Une utilisateur peur appartenir à plusieurs groupes et
    plusieurs utilisateurs peuvent appartenir au meme groupe
  \item Ces utilsateur et ces groupe sont géré avec leur UID et
    leur GID, /etc/passwd et /etc/group s'occuppent de faire la
    correspondances avec les nom lors de l'affichage
  \item \emph{root} (UID 0) est l'utilisateur privilègié
  \item Certaine fonction ne peuvent être executé que par root (par
    exemple: modifier l'horloge, la configuration réseau)
  \item Il existe plusieurs manières pour devenir \c{root}
    \begin{itemize}
    \item \c{login}
    \item \c{su}
    \item \c{sudo}
    \end{itemize}
  \item Chaque fichier du système est associé à un propriétaire et à un groupe
  \item Chaque fichier possède 3 + 3 x 3 droits (modifiable par \emph{chmod(1)}):
    \begin{itemize}
    \item Ecriture (w)
    \item Lecture (r): Pour un répertoire, cela permet de lister le contenu
    \item Execution (x): Pour un répertoire, cela permet d'accéder au
      fichiers contenus dans le répertoire. Pour un fichier cela permet de
      l'éxecuter
    \end{itemize}
  \item Ces trois droits sont répétés pour
    \begin{itemize}
    \item Le propriétaire (u)
    \item Le groupe (g)
    \item Le reste du monde (o)
    \end{itemize}
    \begin{lstlisting}
$ ls -ld /bin/ls ~/.bashrc /tmp /bin/ping
-rwxr-xr-x  1 root root 105840 Apr  1 05:09 /bin/ls
-rwxrwxrwx  1 jezz jezz     16 Feb 21 23:03 /home/jezz3/.bashrc
-rwsr-xr-x  1 root root  35712 Nov  8  2011 /bin/ping
drwxrwxrwt 13 root root   4096 Jul 28 20:42 /tmp
   \end{lstlisting}
 \item Il est pratique d'écrire ses droit sous la forme octal: 755, 644, etc...
 \item Il existe des système de gestion de droits plus fin (Access Control List), mais pas aussi utilisé.
   \item Il existe en plus:
\begin{itemize}
  \item SetUid/SetGid (comme ping): La commande prend les droits du propriétaire/groupe lorsqu'elle s'éxecute
  \item Sticky Bit (comme /tmp): Il est possible de créer des fichier mais, seul le propriétaire du fichier peut l'effacer.
\end{itemize}
\end{itemize}
\end{frame}

\begin{frame}[fragile=singleslide]{Gestion des paquets}
  \begin{itemize}
  \item La principale raison obligeant à être root pour installer un
    paquet est pour écrire les répertoires appartenant à root.
  \item Un programme est livré sous forme d'un paquet contenant: la
    binire, la doc et les éventuelles ressources
  \item Il s'agit ni plus d'une tarball (tar est utilisé dans les .deb
    alors que cpio est utilisé dans les .rpm).
  \item Lors de la decompression (\c{dpkg -i}, \c{rpm -i}, \c{ipk
      -i}), on écrit dans une base de donnée les nom des fichiers
    décompressé afin de pouvoir facilement supprimer le paquet
  \item On ajoute un fichier normalisé à cette tarball afin d'avoir
    des information supplémentaires:
    \begin{itemize}
    \item Version
    \item Description
    \item Dépendances
    \item Signature
    \item ...
    \end{itemize}
  \item On peut interroger la base de donnée des fichier installés
    avec (\c{dpkg -l}, \c{dpkg -L}, \c{dpkg -S}, \c{rpm -q})
  \item Afin de ne pas devoir récupérer chaque paquets manuellement,
    on crée des dépots centralisé et indexé. Des outils peuvent
    interroger ces base (apt-get, urpmi, yum, ...). il peuvent
    automatiquement recupérer les dépendance et appeller dpkg/rpm
  \item Lors de l'installation, on note si le paquet a été demandé
    explicitement ou si il a été installé à cause d'une
    dépendance. Lorsqu'un paquet n'a plus d'utilité, on peut le
    désinstaler automatiquement.
  \item Pleins d'autres outils utiles: \c{apt-cache}, \c{apt-get source},
    \c{apt-get source -b}, \c{apt-get builddeps}, etc...
  \end{itemize}
\end{frame}

